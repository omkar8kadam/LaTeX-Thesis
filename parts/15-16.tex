\documentclass[12pt]{report}
\usepackage[utf8]{inputenc}
\usepackage{graphicx}

\begin{document}
\par
Equation (2.19) means the contribution in $J(H_M)$ of the $i$th example is the error of $h_M(x)$ on $x_i, e^{-y_ia_Mh_M(x_i)}$, multiplied by $e^{-y_iH_{(M-1)}(x_i)}$. So we can define the weight of the $i$th example after the $M$th iteration as:
\begin{equation}
	w^{(M)}(x_i) = e^{-y_iH_M(x_i)}
	\label{eqn:2.20}
\end{equation}
\par
	Intuitively this is saying if the current strong classifier already can classify an example $x_i$ correctly, then it doesn’t really matter what performance the new weak classifier can achieve on $x_i$; on the other hand, we can expect $h_M(x)$ to fix this problem.
\par
	
So the updated formula of $w^{(m)}(x_i)$ is:

\begin{equation}
	w^{(M)}(x_i) = w^{(M-1)}(x_i)e^{-y_ia_Mh_M(x_i)}
	\label{eqn:2.21}
\end{equation}

\par	
Usually, the training set is unbalanced—the number of training images with $y_i$= 1 is larger(smaller) than the number of images with $y_i$=$-1$. In such cases, the $y_i$=1 ($y_i$=$-1$) set will dominate the training process. To eliminate this assymetry, one way is to initialize the sample weights such that:
\begin{equation}
	\sum_{i}w_i^{(0)} y_i=0
	\label{eqn:2.23}
\end{equation}
\par
The algorithm is summarized in \ref{alg:1}.
			
\section{Face Alignment}
The aim of face alignment is to achieve more accurate localization of the face, and usually the facial components (feature points). A typical result of face detection and face alignment is compared in Fig.2.7. As we can see, detection considers the images in terms of areas whereas alignment has a precision of pixels.
\par
Various algorithms have been proposed since 1990's. Gu \textit{ et al}. \cite{29} use histogram information to localize mouth corners and eye corners. In \cite{26}, Ian and Marian implement a preliminary alignment system using Gabor filter to detect pupils and philtrum. Curve fitting algorithms, especially Active Shape Model \cite{18} and its offspring may be the most successful alignment methods nowadays. Cootes \textit{ et al}. \cite{18} \cite{45} \cite{16} propose Active Shape Model and apply it to face image. After that, ASM is widely used in face image processing; a great amount of effort has been made to improve its speed, accuracy and robustness. In \cite{36} the Active Shape Model is combined with the Gabor filter; Li \textit{ et al}. \cite{78} propose Direct Appearance Model; \cite{56} and \cite{101} enhance ASM by using 2-D local textures feature for local search. 

\par
This section will mainly focus on curve-fitting types of method, in particular Active Shape Model and its ancestor Active Contour Model.

\begin{algorithm}[t]
	\KwInput\:{Training examples $ Z={(x1,y1),...,(xn,yn)}$, where $N=a+b$, of which $a$ examples have $y_i=1$ and $b$ examples have $y_i=-1$ \\
	The number $M$ of weak classifiers to be combined.}

	\KwInitialization\:{
		\[
			$$ w_i^{(0)} = 
			\begin{cases}
				$\frac{1}{2a}$ &\quad \text{for those examples with y_i = 1}\\ 
				$\frac{1}{2b}$ &\quad \text{otherwise}
			\end{cases}
			$$
		\]
	}
\end{algorithm}

\textbf{Initialization:} \\
\par $w_i^{0}$=
$\begin{cases}
	\frac{1}{2a}  & \quad \text{for those examples with yi=1} \\
	\frac{1}{2b}  & \quad \text{otherwise}
\end{cases}$ \\
\par end

\textbf{Forward Inclusion:} \\

 for m=1, ..., M do \\
\par \{
\par choose optimal $h_M$ to minimize the weighted error \\
\par choose $a_m$ according to (2.17) \\
\par update $w_i^{m} \leftarrow w_i^{(M-1)}e^{-y_ia_Mh_M(x_i)}$ and normalize to $\displaystyle\sum_{i}w_i^{(m)}=1$  \\
\par \}
\par end \\
 end \\
 \par 
\textbf{Output:} \\
\par Classification function: $H_M(x)$ as in 2.20 \\
\par Class label prediction:$y\land(x)$=sgn$(H_M(x))$

\begin{thebibliography}{9}
	\bibitem{16}
	T. Cootes, C. Taylor, and A. Lanitis. Multi-resolution search with active shape models. pages A:610–612, 1994. 
	
	\bibitem{18}
	T. F. Cootes, C. J. Taylor, D. H. Cooper, and J. Graham. Active shape models-their training and application. Comput. Vis. Image Underst., 61(1):38–59, 1995.
	
	\bibitem{26}
	I. Fasel, M. Bartlett, and J. Movellan. A comparison of gabor filter methods for automatic detection of facial landmarks, 2002. 
	 
	\bibitem{29}
	H. Gu, G. Su, and C. Du. Feature points extraction from faces. Proceedings of Image and Vision Computing New Zealand Conference, pages 154–158, November 2003. 
	
	\bibitem{35}
	A. K. Jain and S. Z. Li. Handbook of Face Recognition. Springer-Verlag New York, Inc., Secaucus, NJ, USA, 2005. 
	
	\bibitem{36}
	F. Jiao, S. Li, H.-Y. Shum, and D. Schuurmans. Face alignment using statistical models and wavelet features. cvpr, 01:321, 2003.
	
	\bibitem{45}
	A. Lanitis, C. J. Taylor, and T. F. Cootes. Automatic interpretation and coding of face images using flexible models. IEEE Trans. Pattern Anal. Mach. Intell., 19(7):743–756, 1997. 
	
	\bibitem{56}
	Y. Liu, Y. Li, L. Tao, and G. Xu. Multi-view face alignment guided by several facial feature points. In ICIG ’04: Proceedings of the Third International Conference on Image and Graphics (ICIG’04), pages 238–241, Washington, DC, USA, 2004. IEEE Computer Society. 
		
	\bibitem{78}
	Y. ShuiCheng and Q. Cheng. Multi-view face alignment using direct appearance models. In FGR ’02: Proceedings of the Fifth IEEE International Conference on Automatic Face and Gesture Recognition, page 324, Washington, DC, USA, 2002. IEEE Computer Society. 
		
	\bibitem{101}
	S. Xin and H. Ai. Face alignment under various poses and expressions. In Tao et al. [85], pages 40–47. 
\end{thebibliography}
\end{document}